\documentclass[sigconf]{acmart}

% Remove ACM reference format
\settopmatter{printacmref=false}
\renewcommand\footnotetextcopyrightpermission[1]{}
\pagestyle{empty}

% Additional packages
\usepackage{graphicx}
\usepackage{booktabs}
\usepackage{multirow}
\usepackage{amsmath}
\usepackage{algorithm}
\usepackage{algorithmic}
\usepackage{url}

% Custom commands
\newcommand{\todo}[1]{\textcolor{red}{TODO: #1}}
\newcommand{\fixme}[1]{\textcolor{blue}{FIXME: #1}}

\begin{document}

\title{Your High-Risk AI in Healthcare Project Title}

\author{Your Name}
\email{your.email@university.edu}
\affiliation{%
  \institution{Your University}
  \city{Your City}
  \state{Your State}
  \country{Your Country}
}

% Add co-authors if working in a team
% \author{Co-author Name}
% \email{coauthor.email@university.edu}
% \affiliation{%
%   \institution{Your University}
%   \city{Your City}
%   \state{Your State}
%   \country{Your Country}
% }

\begin{abstract}
This paper presents our high-risk approach to [specific healthcare problem]. 
We propose [your novel method] that [brief description of innovation]. 
Our experimental results on [dataset name] demonstrate [key findings]. 
Despite the high-risk nature of our approach, we provide valuable insights into [what you learned].
\end{abstract}

\maketitle

\section{Introduction}
\label{sec:introduction}

Healthcare represents one of the most critical domains where artificial intelligence can have transformative impact. 
[Provide context about the specific healthcare problem you're addressing]

\subsection{Motivation}
[Explain why this problem is important in healthcare]

\subsection{Problem Statement}
[Clearly state the specific problem you're trying to solve]

\subsection{High-Risk Innovation}
[Describe your novel approach and why it's considered "high-risk"]

\subsection{Contributions}
This work makes the following contributions:
\begin{itemize}
    \item [Contribution 1]
    \item [Contribution 2]
    \item [Contribution 3]
\end{itemize}

\section{Related Work}
\label{sec:related_work}

\subsection{Traditional Approaches}
[Discuss existing methods for your problem]

\subsection{Recent Advances}
[Discuss recent work in the area]

\subsection{Research Gap}
[Explain what's missing and why your approach is needed]

\section{Methodology}
\label{sec:methodology}

\subsection{Problem Formulation}
[Formally define your problem]

\subsection{Dataset}
[Describe your dataset, including size, features, and preprocessing steps]

\subsection{Baseline Method}
[Describe the standard approach you're comparing against]

\subsection{Proposed Method}
[Describe your high-risk approach in detail]

\subsection{Evaluation Metrics}
[Define how you measure success]

\section{Experimental Setup}
\label{sec:experimental_setup}

\subsection{Implementation Details}
[Describe your implementation, including frameworks, hardware, etc.]

\subsection{Hyperparameters}
[List key hyperparameters and how they were chosen]

\subsection{Computational Resources}
[Describe computational requirements and resources used]

\section{Results}
\label{sec:results}

\subsection{Quantitative Results}
[Present your main results with tables and figures]

\subsection{Qualitative Analysis}
[Provide qualitative insights and error analysis]

\subsection{Comparison with Baseline}
[Compare your method with the baseline]

\section{Discussion}
\label{sec:discussion}

\subsection{Key Findings}
[Discuss what you discovered]

\subsection{Limitations}
[Honestly discuss limitations and failures]

\subsection{Ethical Considerations}
[Address ethical implications of your work]

\section{Conclusion and Future Work}
\label{sec:conclusion}

\subsection{Summary}
[Summarize your work and its impact]

\subsection{Future Directions}
[Discuss what you would do differently and next steps]

\subsection{Lessons Learned}
[Reflect on what you learned from this high-risk project]

\section*{Acknowledgments}
[Thank your team members, instructor, and any resources you used]

\bibliographystyle{ACM-Reference-Format}
\bibliography{references}

\end{document} 